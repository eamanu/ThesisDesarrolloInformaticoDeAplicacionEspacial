No es un secreto de que llevar a cabo una misión satelital representa grandes costos
y largos periódos de tiempo de desarrollo de la misión.
La NASA también padeció estos problemas hace algunas décadas.
Pero esta, al ser una organización consciente de sus habilidades (y sobre todo con recursos),
se puede dar el lujo de innovar, de ir en contra de las corriente y filosofías normales de desarrollo.
Así es como surge su lema \textit{Faster, Better and Cheaper} (FBC).
Este lema fue introducido por el administrador de la NASA Dan Goldin a
principios de la decáda de los 90's. Dan Goldin tiene una basta experiencia en
el desarrollo de pequeños satélites. Él estaba convencido de que el desarrollo
``clásico'' de misiones satelitales de NASA necesitaba un cambio de enfoque,
para lograr reducir el tiempo de desarrollo, y pasar de escalas de tiempo de
décadas a años y sobre todo lograr reducir costos.

Es sabido que este lema produce una gran cantidad de fallas y
cadencias en las misiones satelitales \citep{Paxton07}. Sin embargo,
esta cadencia es lo que impulsa la innovación, y el desarrollo y entrenamiento de nuevos
managers, ingenieros y científicos, los cuales serán los líderes de las
nuevas generaciones. \citep{Paxton07}. 

