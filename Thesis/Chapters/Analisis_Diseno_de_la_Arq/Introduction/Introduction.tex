\section{Introducción}
Los proyectos aeroespaciales emplean una arquitectura de aviónica que se denomina \textbf{federada}, en la cual cada computadora del sistema se diseña para que desarrolle una sola función específica \citep{Loveless15}. Esta estrategia de diseño tiene varias ventajas, por tal motivo ha sido utilizada a lo largo de los años. En contraposición, cuenta con varias desventajas, que alientan al surgimiento de nuevas formas de pensamiento y desarrollo de aviónica de sistemas espaciales. Algunas de estas desventajas que ya fueron mencionadas con anterioridad son la masa y una utilización ineficiente de los procesadores. Para ello se está comenzando a desarrollar arquitecturas con el paradigma IMA.
