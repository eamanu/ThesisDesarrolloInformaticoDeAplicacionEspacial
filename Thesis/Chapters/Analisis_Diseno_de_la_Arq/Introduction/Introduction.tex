\section{Introducción}
\begin{comment}
Los proyectos aeroespaciales emplean una arquitectura de aviónica que se denomina \textbf{federada}, en la cual cada computadora del sistema se diseña para que desarrolle una sola función específica \citep{Loveless15}. Esta estrategia de diseño tiene varias ventajas, por tal motivo ha sido utilizada a lo largo de los años. En contraposición, cuenta con varias desventajas, que alientan al surgimiento de nuevas formas de pensamiento y desarrollo de aviónica de sistemas espaciales. Algunas de estas desventajas que ya fueron mencionadas con anterioridad son la masa y una utilización ineficiente de los procesadores. Para ello se está comenzando a desarrollar arquitecturas con el paradigma IMA.
\end{comment}

En este capítulo, en primer lugar, se plantea una serie de requerimientos (no estrictos) que ayudan en el desarrollo de una arquitectura para una misión satelital ficticia basada en componentes COTS. Luego, en base a estos requerimientos, se presentan modelos para la medición de confiabilidad de tres tipos diferentes de topologías de comunicación de subsistemas satelital. Con ello, se pudo elegir cual es la topología que asegura una mayor tolerancia a fallas.

Como siguiente paso 

