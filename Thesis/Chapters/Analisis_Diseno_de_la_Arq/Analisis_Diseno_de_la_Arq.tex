\vspace{1cm}

\itshape
En este capítulo se lleva a cabo un estudio de la confiabilidad de tres tipos
de topologías tolerantes a fallas (Sección \ref{seccion:TopologiaEstudio}
página \pageref{seccion:TopologiaEstudio}), candidatas a ser utilizada en el
diseño de la arquitectura propuesta en este trabajo de tesis.

Para ello, en la Sección \ref{sec:req_analisis} (página \pageref{sec:req_analisis})
se realiza una descripción de algunos requerimientos (no estrictos), que guiarán
el desarrollo de una arquitectura para un misión satelital ficticia basada en
componentes COTS. En base a estos ``requerimientos'', se presentan los modelos para
la medición de confiabilidad de las topologías mencionadas en el párrafo anterior. 

En la Sección \ref{sec:protocolo_communicacion} (página \pageref{sec:protocolo_communicacion})
se describe resumídamente el protocolo CANae 0.1 Alpha, que está basado en CAN y
que fue desarrollado en este trabajo de tesis. En el apéndice \ref{Appendix:A}
se describe este protocolo con mayor detalle
\upshape

\noindent\rule{\textwidth}{2pt}

\vspace{1cm}
% section that make an introduction
%\section{Introducción}
Los proyectos aeroespaciales emplean una arquitectura de aviónica que se denomina \textbf{federada}, en la cual cada computadora del sistema se diseña para que desarrolle una sola función específica \citep{Loveless15}. Esta estrategia de diseño tiene varias ventajas, por tal motivo ha sido utilizada a lo largo de los años. En contraposición, cuenta con varias desventajas, que alientan al surgimiento de nuevas formas de pensamiento y desarrollo de aviónica de sistemas espaciales. Algunas de estas desventajas que ya fueron mencionadas con anterioridad son la masa y una utilización ineficiente de los procesadores. Para ello se está comenzando a desarrollar arquitecturas con el paradigma IMA.

% section that talk about requeriments
\section{Requerimientos para el análisis}
En esta sección no se pretende realizar una lista de requerimientos formales para el desarrollo de una arquitectura, esto se hace el capítulo siguiente. El objetivo de esta sección, es llevar a cabo una guía sencilla de las partes principales de un sistema satelital. Con esto en mente se podrá desarrollar diferentes topologías de comunicación, y luego analizar su tolerancia a falla.

Se supondrá una misión de 15 años (sin carga útil para simplificar el análisis) cuyo sistema estará compuesto por los siguientes subsistemas (en inglés para mantener correspondencia con la literatura) basado en \cite{Fontescue03}:
\begin{itemize}
\item Power Subsystem
\item Atitude and Orbit control Subsystem
\item Telemetry and Subsystem
\item Thermal Subsystem
\item Propulsion Subsystem
\item Data Handlig Subsystem
\end{itemize}

El sistema, entonces, está compuesto por 6 nodos. Los nodos se suponen computadoras con capacidad de procesamiento suficiente para cada subsistema. Cada una de estas computadoras/nodos es un componente COTS con un cierto grado de confiabilidad (se suponen lo suficientemente bajo como para no ser utilizado en forma directa en el desarrollo de satélites). A nivel de \ac{HW} estos nodos cuenta con tolerancia a fallas, lo que aumenta su confiabilidad. El subsistema de Data Handling tiene que tener comunicación con todos los subsistemas, ya que será la encargada de controlar el correcto funcionamiento del sistema, enviar comandos, y empaquetar telemetría de los sensores.




% section that talk about the nomenclautre
\section{Nomenclatura}
Durante el análisis se utilizará la siguiente nomenclatura:

$T_m$ \tab Tiempo de la misión
$TTNF$ \tab Tiempo hasta la siguiente falla
$\lambda$ \tab Tasa de falla
$MTBF$ \tab Tiempo medio entre fallas
$MTTR$ \tab Tiempo medio de reperacion
$A(t)$ \tab Disponibilidad
$R(t)$ \tab Confiabilidad

Como se definió anteriormente el $T_m$ es de 15 años. Para simplificar los trabajos de cálculos y ejecución del presente trabajo, se supone que la arquitectura puede fallar solo una vez durante la misión satelital. Por lo tanto, el $TTNF$ será de 15 años. La tasa de falla se define como: $$\lambda = \frac{1}{15}$$. Suponiendo que la arquitectura no debería fallar durante los 15 años de misiones se da la siguiente sisituación: $$MTTBF = MTTR = 15 $$ Por otro lado, La disponibilidad es $A(t) = 99\%$. La confibialidad, como se estudió en secciones anteriores, es $R(t) = e^{- \lambda t}$



% section that talk about the differents topology
\section{Estudio de topologías de arquitecturas}\label{seccion:TopologiaEstudio}
Luego de un estudio exhaustivo del estado de la cuestión (\autoref{chap:estado_del_arte} Página \pageref{chap:estado_del_arte}) se llegó a la conclusión de que las topologías más estudiadas, por ende más maduras y sencillas de aplicar a las actividades que se pretenden realizar en la presente tesis son:
\begin{itemize}
  \item Árbol binario
  \item Red distribuida
  \item Arquitectura hypercube
\end{itemize}

En esta sección, se definirá modelos para medir la confiabilidad de cada una de las topologías de arquitectura que aseguren una mayor tolerancia a fallas a nivel de sistema, de modo tal que si se llega a producir una falla en cualquiera de los nodos (sistemas de procesamiento) de la arquitectura, esta puede reconfigurarse, pemitiendo que esta continúe funcionando, sin sufrir ningún tipo de degradación, aún en la presencia de fallas.

\subsection{Árbol binario}
En primer lugar se planteó un árbol binario de cuatro niveles con backup, basandose en el diseño de \cite{Raghavendra84} \ref{fig:Reliability_binary_tree_4_levels}. Este diseño cuenta con $2^n - 1 =  15$ nodos. De lo estudiado en \autoref{sec:binary_tree} la confiabilidad puede ser calculada de la siguiente manera: $$R_{sys} = R^{2n +1} \prod_{k=0}^{n-1}{[(2^kc+1) - 2^kcR]}$$

Con esto se puede observar la confiabilidad de la red con respecto al tiempo en la Figura \ref{fig:Reliability_binary_tree_4_levels_2}.

\begin{figure}[H]
 \centering
 \includegraphics[scale=0.5]{images/Capitulo4/Reliability_BinaryTree.png}
  \caption{Confiabilidad con respecto al tiempo de árbol binario de 4 niveles}
\label{fig:Reliability_binary_tree_4_levels_2}
\end{figure}

La cantidad de niveles que se eligió para esta topología depende del número de subsistemas que se requieren.

Los nodos backup se mantienen inactivos, es decir no participan en el procesamiento durante la vida normal del sistema. En caso de producirse una falla, se supone que estos nodos de redundancia, comienzan a funcionar automáticamente, sin ningún tipo de retraso. Esto, que no corresponde con la realidad, permite simplificar los cálculos para este trabajo de tesis.


\subsection{Red distribuida}
El estudio de una topología de red distribuida no es tan sencilla como la que se plantea para un árbol binario. Para el desarrollo de esta red, además de los 6 nodos que representa cada uno de los subsistemas requeridos, se agregan 2 nodos de redundancia (este este caso se opta por nodos de redundancias activos). Por lo tanto se tiene una red de 8 nodos. Siguiendo la metodología de desarrollo presentado por \cite{Pradhan82}, se llevó a cabo la red que se presenta en la Figura \ref{fig:distributed_net}.

\begin{figure}[h]
 \centering
 \includegraphics[scale=0.75]{images/Capitulo4/distributed_network.jpg}
  \caption{Red distribuida}
\label{fig:distributed_net}
\end{figure}

Esta cuenta con 4 nodos de grado 4, 2 nodos de grado 3, y 2 nodos de grado 2. Ante cualquier falla de alguno de los nodos de la red, esta topología tiene la capacidad de formar subredes, que mantienen todos los nodos conectados (a excepción del fallado), asegurandose la funcionalidad y reconfiguración del sistema completo. Estratégicamente y para lograr la condición mencionada anteriormente, se crea la \textbf{condición} de que pueden fallar hasta 4 nodos en simultaneo. Esto aseguraría de que la red continuará funcionando aún en la presencia de fallas (definición de tolerancia a fallas).

Se modificó la fórmula desarrollada  por \cite{Stivaros92}, para lograr una coherencia entre los modelos que se plantean en el presente trabajo. Teniendo en cuenta que la confiabilidad del sistema completo es: $$R(t) = \prod_{v \in S} e^{- \lambda t}$$
donde $v$ representa el nodo y $S$ es el subsistema funcional. Es decir, que este modelo recorre todos los nodos funcionales. Cuando existen nodos con fallas, y que dejan de ser funcionales, el modelo es el siguiente: $$R_{sys} =\sum_{i=0}^{k} (( \prod_{v \in S} R(t)) - (\prod_{v \notin S} (1 -  R(t) c )))$$
donde c se definió como una \textit{constante de degradación del sistema} para modelar una degradación del sistema.

En la Figura \ref{fig:reliability_distributedNet} se puede observar la \textit{curva de confiabilidad} del sistema, para el caso en el que todos los nodos se encuentren funcionales.

\begin{figure}[H]
 \centering
 \includegraphics[scale=0.5]{images/Capitulo4/Reliability_DistributedNet.png}
  \caption{Confiabilidad de red distribuida}
\label{fig:reliability_distributedNet}
\end{figure}


Para el caso de la falla de todos los nodos la \textit{curva de confiabilidad} (Figura \ref{fig:reliability_distributedNet_4Nodes_Fail}) muestra correctamente la degradación esperada de la confiabilidad, con respecto al sistema funcionando correctamente sin ninguna falla.

\begin{figure}[H]
 \centering
 \includegraphics[scale=0.5]{images/Capitulo4/Reliability_DistributedNet_4Nodes_Fail_2.png}
  \caption{Confiabilidad de red distribuida con 4 nodos fallando}
\label{fig:reliability_distributedNet_4Nodes_Fail}
\end{figure}

\subsection{Red hypercube}
Para el caso de la red hypercube se llevaron a cabo modificaciones al modelo planteado por \cite{Mostafa14}, con el propósito de mantener coherencia en los modelos y cálculos que se realizan en este trabajo. Se diseñó una red 3-dimensional, con 8 nodos,  de los cuales 6 nodos corresponden a los diferentes subsistemas y 2 nodos son utilizados como redundancias activas (Figura \ref{fig:Reliability_Hypercube}).

\begin{figure}[H]
 \centering
 \includegraphics[scale=0.75]{images/Capitulo4/Hypercube2.jpg}
  \caption{Red Hypercube}
\label{fig:Hypercube}
\end{figure}

En este trabajo de tesis, la confiabilidad se calculó por medio del siguiente modelo: $$R_{sys} = 1- [N (1 - e^{- \lambda t}]$$

Como se puede observar el modelo es similar al modelo de árboles binario. Como se pudo estudiar en la bibliografía, árboles binarios y redes hypercube presentan varias similitudes, incluso se emebebe árboles binarios en este tipo de red. La \textit{curva de confiabilidad} de esta red se observa en la Figura \ref{fig:Reliability_Hypercube}

\begin{figure}[H]
 \centering
 \includegraphics[scale=0.5]{images/Capitulo4/Reliability_Hypercube.png}
  \caption{Confiabilidad de red hypercube}
\label{fig:Reliability_Hypercube}
\end{figure}

\section{Topología utilizada en la arquitectura a diseñar}

Teniendo en cuenta los modelos presentados anteriormente, se procede a realizar una comparación de las diferentes curvas. El resultado de esto, permitirá conocer de manera analítica qué topología presenta una mayor tolerancia a fallas através del tiempo. Además, teniendo en cuenta lo estudiado en el estado de la cuestión, se puede realizar una comparación conceptual de las tres topologías.

La aplicación de árboles binario podría resultar una buena opción, ya que representa un desarrollo sencillo. En contraposición se puede indicar que existe un alto grado riesgo de que se produzca una falla en el nodo raíz y de su redundancia, lo cual pondría en peligro la misión. Así mismo, presenta otro punto negativo que se puede mencionar y es la gran cantidad de enlaces que esta necesita para mantener a todos los nodos de la red conectados.

Por otra parte, la red distribuida cuenta con la capacidad de distribuir el trabajo en todos sus nodos. Esto quiere decir, que si se produce una falla irrecuperable en cualquiera de sus nodos, la arquitectura podría continuar funcionando sin verse afectada por la ausencia de dicho nodo. Esto demanda un procesamiento computacional extra, y la necesidad de desarrollar algoritmos de ruteo especiales. Además, como punto negativo se puede mencionar que también, al igual que los árboles binarios, necesitan una gran cantidad de enlaces.

Por último, las topologías hypercube tienen un excelente respaldo teórico, exigen menor cantidad de enlaces, y pueden tolerar la falla de una gran cantidad de nodos (hasta el 50\% de los nodos). Su complejidad aumenta en gran medida, cuando se desarrollan arquitecturas de más dimensiones, lo cual también incrementa su confiabilidad.

Como se mencionó en el primer párrafo, se realizó una comparación de modelos de confiabilidad de las tres topologías estudiadas. Se asumió que la distribución de la confiabilidad es exponencial, con una tasa de falla fija de 1/15, y se estudió su evolución en un rango $[0,1]t$. El resultado de esta comparación se la puede observar en la Figura \ref{fig:comparative_reliablities} y en la Tabla \ref{table_comprative_reliability}.

Se observa que tanto las redes distribuidas como la hypercube presentan una mayor confiabilidad a través del tiempo que  los árboles binarios. Si bien, las redes distribuidas y la hypercube tienen una curva similar, la primera presenta una mayor confiabilidad sostenida en el tiempo.

\begin{figure}[H]
 \centering
 \includegraphics[scale=0.5]{images/Capitulo4/comparative_reliablities.png}
  \caption{Comparación de confiabilidad}
\label{fig:comparative_reliablities}
\end{figure}

\begin{table}[H]
\centering
\caption{Comparación de confiabilidad de topologías}
\label{table_comprative_reliability}
\begin{tabular}{r|r|r|r|}
\cline{2-4}
\multicolumn{1}{l|}{} & \multicolumn{3}{c|}{Topologías de red} \\ \hline
\multicolumn{1}{|c|}{T} & \multicolumn{1}{c|}{Tree Net} & \multicolumn{1}{c|}{Distr Net} & \multicolumn{1}{c|}{Hyper Net} \\ \hline
\multicolumn{1}{|r|}{0} & 1 & 1 & 1 \\ \hline
\multicolumn{1}{|r|}{0.001} & 0.803766 & 0.999947 & 0.999947 \\ \hline
\multicolumn{1}{|r|}{0.002} & 0.64604 & 0.999893 & 0.999893 \\ \hline
\multicolumn{1}{|r|}{0.003} & 0.519265 & 0.99984 & 0.99984 \\ \hline
\multicolumn{1}{|r|}{0.004} & 0.417368 & 0.999787 & 0.999787 \\ \hline
\multicolumn{1}{|r|}{0.005} & 0.335466 & 0.999733 & 0.999733 \\ \hline
\multicolumn{1}{|r|}{...} & ... & ... & ... \\ \hline
\multicolumn{1}{|r|}{0.995} & 0 & 0.948317 & 0.947109 \\ \hline
\multicolumn{1}{|r|}{0.996} & 0 & 0.948266 & 0.947056 \\ \hline
\multicolumn{1}{|r|}{0.997} & 0 & 0.948216 & 0.947003 \\ \hline
\multicolumn{1}{|r|}{0.998} & 0 & 0.948165 & 0.94695 \\ \hline
\multicolumn{1}{|r|}{0.999} & 0 & 0.948115 & 0.946897 \\ \hline
\end{tabular}
\end{table}


Con esto se puede concluir que la topología que presenta un mayor grado de confiabilidad es la que responde a una filosofía distribuida (bajo las condiciones en las que fueron estudiadas). Por lo tanto, la arquitectura satelital, tolerante a fallas y basada en componentes COTS que se desarrolla en la presenta tesis se basa en una \textbf{topología distribuida} para interconectar los diferentes subsistemas.

\section{Topología propuesta}
Sobre la base de los resultados presentados en \citep{Arias17} se puede establecer que la topología propuesta (red distribuida) es la más adecuada para el desarrollo de una arquitectura tolerante a fallas como se demuestra en la Figura \ref{fig:topo_propuesta}. En esta se puede observar que cada subsistema (thermal, power, telemetry, etc.) tiene su propia CPU controladora. Estas CPU se conectan a los nodos COTS de la red.

\begin{figure}[h!]
 \centering
 \includegraphics[scale=0.5]{images/Capitulo4/arq_final.jpg}
  \caption{Arquitectura propuesta utilizando topología de red distribuida}
\label{fig:topo_propuesta}
\end{figure}

El modelo exige como requerimiento que cada nodo debe estar compuesto por una computadora (componente COTS) que es la encargada de realizar el procesamiento de las tareas. También, debe exitir un puente de comunicación entre la red y la CPU del subsistema. De este modo se hace frente a posibles fallas en la computadora del nodo. Esta conexión se observa en la Figura \ref{fig:conn_prop}

\begin{figure}[h!]
 \centering
 \includegraphics[scale=0.3]{images/Capitulo4/com_nodo.jpg}
 \caption{Conexión entre la red y el subsistema}
\label{fig:conn_prop}
\end{figure}

% section that talk about the comm protocol
\section{Protocolo de comunicación}
Como se estudió en el marco teórico, en una arquitectura
de comunicación, el protocolo CAN trabaja en las
capas inferiores del modelo de OSI. CAN propone cómo debe ser
el medio físico, y la manera de transmisión de las señales.
Además, preve cuál es la estrategia a utilizar  para sincronizar
la transmisión de datos sin nigún tipo de latencia, ni tampoco
colisiones de mensajes. Así, este protocolo permite que estos
sean enviados teniendo en cuenta sus prioridades, eliminando las colisiones, y
por lo tanto también, el tiempo de espera en la que los nodos se quedan
relegados cuando han tratado de enviar un mensaje al mismo tiempo.

Este protocolo no establece nada sobre ruteos de mensajes, tipos de mensajes y
tipos de datos, que a la hora de desarrollar una arquitectura de aviónica es
importante tener en cuenta esos detalles. Es necesario, que exista una
capa de servicios que facilite la comunicación entre aplicaciones de usuarios
de diferentes nodos, y la comunicación entre las aplicaciones de usuario
y el protocolo CAN (de bajo nivel) en sí.
Además, teniendo en cuenta los resultados obtenidos en las secciones
anteriores, se necesita, diseñar un protocolo de comunicación
basado en redes distribuidas, y que a su vez, esté basado en CAN.

Por tales motivos, surge la necesidad
de diseñar un protocolo de comunicación basado en CAN, que se ``monte'' sobre
las capas superiores del modelo de referencia de OSI; y que permita
la distribución de las tareas y el procesamiento llevado a cabo
por los nodos. 

El protocolo CANae se encuentra en su primera versión 0.1 Alpha. Esto hace
referencia a que en esta instancia de trabajo se trata de comprender
el problema y llevar a cabo un diseño preliminar del protocolo. De esta
manera se podrá extraer, tanto puntos positivos, como puntos negativos; o
más bien fortalezas y debilidades del stack de servicios que brinda CANae.

CANae actúa en la capa de aplicación del modelo de referencia de OSI. CANae
divide esta capa en dos subcapas:
\begin{itemize}
\item CANae Application Layer
\item CANae High Application layer
\end{itemize}

\begin{figure}[h!]
 \centering
 \includegraphics[scale=0.5]{images/Secciones/AppendixA/CANAE.JPG}
  \caption{Estructura de la capa de aplicación de CANae en alto nivel}
\label{fig:CANAEC4}
\end{figure}

Esto puede observar se en la Figura \ref{fig:CANAEC4}. En el gráfico se observa que
la CANae application Layer, cuenta con 3 interfaces:
\begin{itemize}
\item \textbf{CMS}: ofrece un ambiente orientado a objecto para diseñar aplicaciones de usuario.
  Esta entidad ofrece variables y eventos, y específica como un módulo pude acceder a
las interfaces de CAN.
\item \textbf{NMT}: ofrece un ambiente orientado a objetos para permitir que un módulo (el
NMT Master) se ocupe de la inicialización y posibles fallas de otros módulos (NMT Slaves).
\item \textbf{DBT}: ofrece el servicio para distribuir dinámicamente el identificador de para los diferentes nodos.
\end{itemize}
Debe destacarse que, el servicio CMS ofrece un ambiente orientado a objeto  para diseñar
aplicaciones de usuario, es decir, que ofrece la posibilidad de modelar el
comportamiento del sistema en forma de objeto. Este servicio ofrece la posibilidad
de crear variables y eventos, según las necesidades de los usuarios, que son
utilizadas para diseñar y especificar como la funcionalidad de un módulo puede
acceder a las interfaces de bajo nivel de CAN. Esto propone un grado de innovación
al tratar al protocolo, a los datos, mensajes y eventos como objetos, permitiendo así
un modelado orientado a objetos. Por tal motivo, este protocolo fue modelado
con SysML (System Modeling Language).
Para mayor detalle de estas interfaces se debe consultar en el apéndice \ref{Appendix:A}.

Agregando, dentro de esta capa existen dos entidades que juegan un papel importante
(consultar apéndice \ref{Appendix:A}) las cuales son:
\begin{itemize}
\item \textbf{Gestor de mensajes}: este se encarga de gestionar los mensajes que son enviados y recibidos desde la red.
Esta entidad debe ser capaz de determinar si el mensajes contiene datos o eventos. Trabaja en conjunto
con el CMS.
\item \textbf{Gestor de nodo}: esta entidad se encarga de llevar el control de los nodos existenten en la red. En este
nodo se encuentra la tabla de ruteo primario y secundario necesarios para la correcta comunicación.
\end{itemize}

La subcapa denominada CAN High Application Layer tiene una función más del lado de la gestión de nodos y tareas.
En esta capa se desarrollan los algoritmos necesarios para realizar el ruteo y la reconfiguración de la red ante
fallas, por lo tanto en la High Application Layer se debe implementar el sistema de FDIR. El protocolo no define ningún
algoritmo de ruteo, por lo que queda para el usuario la definición de los algoritmos.
El protocolo  recomienda desarrollar dos algoritmos, el primario y el secundario.
Para aumentar la tolerancia a fallas se pueden desarrollar más algoritmos
secundarios, y el switch entre algoritmos, puede depender de medidas de perfomance.

Así, queda conformada el protocolo de comunicación CANae como uno de los
principales productos de este trabajo de tesis. 


\vspace{1cm}
\noindent\rule{\textwidth}{2pt}

\textbf{\Large{Resúmen}}

En este capítulo se llevó a cabo un estudio de tres tipos de topologías tolerantes
a fallas que fueron candidatas a ser utilizadas como base en la
arquitectura que se diseñó en este trabajo de tesis. 

Teniendo en cuenta los modelos que se han desarrollado en este capítulo se compararon las diferentes curvas de confiabilidad. El resultado de esto, permitió conocer de manera analítica qué topología presenta una mayor tolerancia a fallas a través del tiempo.

Se asumió que la distribución de la confiabilidad es exponencial, con una tasa de falla fija de 1/15, y se estudió su evolución en un rango $[0,1]t$. El resultado de esta comparación se la puede observar en la Figura \ref{fig:comparative_reliablities} y en la Tabla \ref{table_comprative_reliability}.

Se observa que tanto las redes distribuidas como la hypercube presentan una mayor confiabilidad a través del tiempo que  los árboles binarios. Si bien, las redes distribuidas y la hypercube tienen una curva similar, la primera presenta una mayor confiabilidad sostenida en el tiempo.

Así en este capítulo se llegó a la conclusión de que la topología que presenta un mayor grado de
confiabilidad es la que responde a una filosofía distribuida.  Por lo tanto, la arquitectura satelital, tolerante a fallas y basada en componentes COTS que se desarrolla en la presenta tesis se basa en una \textbf{topología distribuida} para interconectar los diferentes subsistemas.
