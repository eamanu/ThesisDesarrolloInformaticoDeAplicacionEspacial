% esta seccion habla sobre los términos importantes a utilizar a lo largo de la tesis
\section{Terminología}\label{sec:terminologia}
Existe una importante diferencia entre los significados de las palabras falla, error y
avería\footnote{En inglés: fault, error y failure.}, que es importante destacar antes de comenzar
con el desarrollo de este trabajo.

Una \textbf{avería} de sistema ocurre cuando el servicio prestado por el sistema ya no coincide con
las especificaciones del mismo \citep{Hanmer07}. Esto quiere decir que existe un problema que tiene
una consecuencia negativa en el sistema completo, logrando que este ya no logre cumplir con sus
especificaciones. Cuando el sistema no se comporta de la manera que es especificada, este ha
fracasado. Esto significa que lo que se espera de un sistema se encuentra descripto, comúnmente en
especificaciones o requerimientos \citep{Pullum01}.

Para la \cite{IEEE610.12} avería es ``la inhabilitación de una sistema o componente a llevar a
cabo las funciones requeridas en los requerimientos específicos de perfomance del mismo''.

\cite{Hanmer07} ejemplifica averías de sistemas cuando: el sistema se bloquea y se detiene cuando no
debería hacerlo, el sistema calcula un resultado incorrecto, el sistema no está disponible, o el
sistema es incapaz de responder a la interacción con el usuario. Cuando el sistema no hace lo que
debe hacer, el sistema ha fracasado. Las averías son detectados por los usuarios mientras usan el
sistema.

Las averías son causados por los errores. Un \textbf{error} es una parte del estado del sistema
que es susceptible de provocar un avería en el sistema. Un error que afecta al servicio, es una
indicación de que una avería se ha producido \citep{Hanmer07}. Un error se puede propagar, es decir
dar a lugar otros errores \citep{Pullum01}.

\cite{IEEE610.12} define error como ``la diferencia entre un valor computado, observado o medido,
con el valor verdadero, especificado o el teóricamente verdadero''.

Los errores se pueden clasificar en dos tipos: errores de tiempo y valores \citep{Hanmer07}. Los
errores de valores son aquellos que se manifiestan como valores discretos incorrectos o estados del
sistema incorrecto. En cambio, los errores de tiempo pueden incluir aquellos que no cumplen con el total de las tareas.

\cite{Hanmer07} especifica los siguiente casos más comunes de errores:
\begin{itemize}
 \item Timing: existe una falta de sincronización en la comunicación de los procesos.
 \item Bucles infinitos: ejecución de un bucle sin detenerse, esto consume memoria, y la
avería del sistema.
 \item Error de protocolo: errores en el flujo de comunicación ya que no coinciden los
protocolos. Mensajes enviados en formato diferente, en tiempos diferentes, a lugares de sistemas
incorrectos.
 \item Inconsistencia de datos: los errores son diferentes en diferentes lugares.
 \item Sobrecarga de sistema: el sistema es incapaz de hacer frente a la sobrecarga de
actividades a la que es expuesta.
\end{itemize}

La causa adjudicada o la hipótesis de un error es una \textbf{falla}, también llamado ``bugs''. Una
\textbf{falla activa} es aquella que produce un error \citep{Pullum01}. Una falla es un defecto que está
presente en el sistema y que puede causar un error \citep{Hanmer07}. Es la desviación actual de lo
correcto \citep{Hanmer07}.

Según \cite{IEEE610.12} una falla es ``un defecto en un dispositivo de hardware o componente; como
por ejemplo un corto circuito o un cable cortado''. También realiza una segunda definición diciendo
que falla es ``un paso incorrecto, proceso, o definición de dato en un programa de computadora''
\citep{IEEE610.12}. Esta última afirmación es la que se usa en el ámbito de este trabajo.

Algunas fallas introducidas en el \ac{SW} se detallan en \cite{Hanmer07}, lo cual señala que
pueden incluir:
\begin{itemize}
 \item Especificaciones incorrecta de requerimientos
 \item Diseño incorrecto
 \item Errores de programación
\end{itemize}

Entonces, como lo indica \cite{Pullum01} con la tolerancia a fallas, lo que se busca es prevenir la
avería mediante la ``tolerancia'' de fallas, las cuales son detectables cuando un error aparece.
Las fallas son el motivo de errores y los errores son motivos de avería \citep{FTDesign}.

También se suele utilizar el término anomalía en las operaciones de vehículos espaciales para
referirse a comportamientos anómalos o no esperados del sistema \citep{SpaceSystemFailures}

En \cite{FTDesign} se describe un ejemplo para diferenciar correctamente estos conceptos. Se considera
el \ac{SW} de una planta nuclear, en la cual existe una computadora que es responsable de controlar
la temperatura, la presión y demás variables de interés para la seguridad del sistema. Se da el
caso de que uno de los sensores detecta que la turbina principal se encuentra girando a una
velocidad menor a la correcta. Esta falla hace que el sistema envíe una señal para aumentar su
velocidad (error). Esto produce un exceso de velocidad en la turbina, lo cual tiene como
consecuencia que la seguridad mecánica apague la turbina. En esta situación el sistema no está
generando energía. Esto se considera un avería, porque el sistema no está entregando el servicio
según lo establecido por los requerimientos. Pero es un avería salvable.

Otro concepto es el de \textbf{mantenibilidad}, esta es la capacidad de un sistema, bajo condiciones normales, de ser restaurado a un estado en el cual puede realizar sus funciones requeridas, cuando se realiza el mantenimiento \citep{Rausand04}.

En secciones posteriores se ven los conceptos de confiabilidad, disponibilidad y seguridad (Sección \ref{subsec:confiabilidad}, \ref{subsec:disponibilidad}, \ref{subsec:seguridad}, respectivamente).
