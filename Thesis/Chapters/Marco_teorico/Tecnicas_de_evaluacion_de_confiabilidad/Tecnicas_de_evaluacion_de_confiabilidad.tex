\section{Técnica de evaluación de fiabilidad}
La evaluación de la fiabilidad es de suma importancia para el desarrollo de sistemas críticos, ya 
que permite identificar que aspectos del comportamiento del sistema juega un papel importante 
\citep{FTDesign}.

Como lo indica \cite{FTDesign} existen dos enfoques convencionales para evaluar fiabilidad:
\begin{enumerate}
 \item Modelado de un sistema en la fase de diseño. 
 \item Aseguramiento del sistema en la fases finales de desarrollo (testing).
\end{enumerate}

La evaulación de la fiabilidad tiene dos aspectos. En primer lugar se tiene una \textit{evaluación 
cualitativa} que permite identificar, clasificar y medir modos de fallas, o eventos combinacionales 
que puedan provocar una falla. El otro aspecto es la \textit{evaluación cuantitativa}, la cual 
permite evaluar en términos de probabilidad los atributos de la fiabilidad 
(Sección \ref{sec:atributos_de_la_fiabilidad}), disponibilidad, seguridad.

\section{Medidas comunes de fiabilidad}
Las medidas de fiabilidad más comunes son las siguientes: failure rate, tiempo medio a la falla, 
tiempo medio de reparación y tiempo medio entre fallas.

\subsection{Failure rate}
Failure rate $\lambda$ es el número esperardo de fallas por unidad de tiemp \citep{FTDesign}. Es 
usual utilizar la dimensión \textit{fallas/horas}.

Generalmente, $\lambda$ se encuentra a nivel de componente. Para conocer el failure rate del 
sistema completo, se puede realizar (a groso modo) una sumatoria de los $\lambda$ de los 
componentes que integran el sistema. $$\lambda=\sum_{i=1}^{n} \lambda_i$$

