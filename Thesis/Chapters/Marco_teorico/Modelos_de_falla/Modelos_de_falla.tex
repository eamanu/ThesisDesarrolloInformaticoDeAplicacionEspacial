\section{Modelos de falla}\label{sec:modelos_fallas}
En esta sección se explica con un poco más de detalle algunos conceptos que se presentaron en secciones anteriores. Los conceptos que se detallarán son los siguientes:

\begin{itemize}
  \item Función de confiabilidad $R(t)$
  \item Función de tasa de falla $\lambda$
  \item Tiempo medio hasta la falla (MTTF)
  \item Vida residual media (MRL)
\end{itemize}

También existen varias distribuciones de probabilidad que son utilizados para modelar el tiempo de vida de componentes de sistemas. Algunos de estos modelos de distribución del tiempo de vida son los siguientes:

\begin{itemize}
  \item Distribución binomial
  \item Distribución exponencial
  \item Distribución gamma
  \item Distribución Weibull
  \item Distribución normal
  \item Distribución Birnbaum-Saunders
  \item Distribución inversa de Gauss
\end{itemize}

\subsection{Tiempo hasta la falla}
El tiempo hasta la falla (T) de un componente del sistema, es el lapso de tiempo desde que el componente empieza a operar hasta la primera falla que se produzca \citep{Rausand04}.

\textbf{Nota: } Se asume que el componente que falla, no puede ser recuperado. Por otro lado, si un sistema falla, puede recuperarse (a través de tecnicas de reconfiguración, componentes back-up).

$T$ no significa que es solo una medida de tiempo. $T$ puede significar \citep{Rausand04}:
\begin{itemize}
  \item Cantidad de kilómetros de un automóvil
  \item Número de rotación de un motor
  \item Número de ciclos de trabajo
\end{itemize}

Debe destacarse que esta no solo es una variable discreta. Una variable discreta puede ser aproximada a una variable continua. Suponiendo que $T$ es una distribución continua con densidad de probabilidad $f(t)$ y una función de distribución: $$F(t) = P(T\leqt) = \int_0^t f(u)du \text{ para t > 0}$$

$F(t)$ demuestra la probabilidad de que un componente falle dentro de un intervalo de tiempo $(0,t]$ \citep{Rausand04}.

La función de densidad de probabilidad $f(t)$ se define como sigue \citep{Rausand04}: $$f(t) = \frac{d}{dt}F(t) = \lim_{\Delta t->0}\frac{F(t+\Delta t) - F(t)}{\Delta t} = \lim_{\Delta t ->0} \frac{P(t<T\leq t + \Delta t)}{\Delta t}$$

Esto indica que para un $\Delta t$ pequeño, $$P(t < T \leq t + \Delta t) \approx f(t)\Delta t$$

\subsection{Función de confiabilidad}
La función de confiabilidad se define como sigue: $$R(t) =  1 - F(t) = P(T>t)$$ para $t>0$.

\subsection{Tasa de falla}
La probabilidad de que un componente falle en un intervalo de tiempo $(t, t+\Delta t]$ dado que el componente ha funcionado hasta t, se tiene: $$P(t < T \leq t + \Delta t | T > t) = \frac{P(t< T \leq t+\Delta)}{P(T>t)} = \frac{F(t+\Delta t) - F(t)}{R(t)}$$

Si se divide esta probabilidad por un intervalo de tiempo $\Delta t$, y haciendo $\Delta t -> 0$, se obtiene la función de tasa de falla.

\subsection{Tiempo medio hasta la falla}
El \ac{MTTF} de un componente está definido por \citep{FTDesign} \citep{Rausand04}: $$MTTF = E(T) = \int_0^\infty tf(t)dt$$

Se demuestra en \cite{Rausand04} que $$MTTF = \int_0^\infty R(t)dt$$

\subsection{Vida restante media}
Considerando un comoponente con tiempo hasta la falla T, que es colocado en operación en el tiempo $t = 0$ y continúa funcionando en el instante t.
La probabilidad de que el componente en edad t sobreviva en un intervalo $x$, entonces se tiene
\citep{Rausand04}: $$R(x|t) = P(T>x+t | T>t) = \frac{P(T>x+t)}{P(T>t)} = \frac{R(x+t)}{R(t)}$$

Entonces: $$MRL(t) = \frac{1}{R(t)}\int_t^\infty R(x)dx$$

Nótese que cuando t = 0, el componente es nuevo, y entonces $\mu(0) = \mu = MTTF$.

\subsection{Distribución binomial}
La distribución binomial es uno de lo más utilizados \citep{Rausand04}. Esta distribución es utilizada en los siguientes casos:
\begin{itemize}
  \item Cuando se tienen n ensayos independientes
  \item Cada ensayo tiene dos posibles resultados
  \item La P(A) = p es la misma en todos los ensayos
\end{itemize}

Entonces $$P(X=x) = {{n}\choose{x}} \cdot p^x (1-p)^{n-x}$$ Donde ${n}\choose{x}$ es el coeficiente binomial y X es el número de n ensayos para alcanzar un resultado A \citep{Rausand04}.

\subsection{Distribución exponencial}
Considerando un componente que entra en operación en el instante $t=0$. El tiempo hasta la falla T del componente tiene una función de densidad de probabilidad de la siguiente forma \cite{Rausand04}:
$$ f(t) =\left \{
\begin{matrix}
  \lambda e^{\lambda t } & \text{para t > 0, } \lambda \text{ > 0}\\
  0                      & \text{para cualquier otro caso}
\end{matrix}
$$

Esta distribución es la que se denomina distribución exponencial con parámetro $\lambda$.

La confiabilidad de esta función  es: $$R(t) = P (T>t) = \int_t^\infty f(u) du  = e^{\lambda t} \text{ para t>0} $$

El \ac{MTTF} es: $$MTTF = \int_0^\infty R(t) dt = \int_0^\infty e^{\lambda t} dt  = \frac{1}{\lambda}$$

La tasa de falla de esta distribución es $\lambda$.

La distribución exponencial es lo que se utiliza más comúnmente en los análisis de confiabilidad.

El \ac{MTTF} es: $$MTTF = \int_0^\infty R(t)dt = \int_0^\infty e^{-\lamda t} = \frac{1}{\lambda}$$

Si hacemos: $$R(MTTF) = R(\frac{1}{\lambda}) = e^{-1}\approx 0.3679$$  se puede calcular la probabilidad de que un componente sobreviva durante el
tiempo medio hasta la falla \citep{Rausand04}.

Mientras que la tasa de falla es: $$z(t) = \frac{f(t)}{R(t)} = \frac{\lambda e^{-\lambda t}}{e^{-\lambda t}} = \lambda $$

\textbf{NOTA IMPORTANTE:} Como se mencionó anteriormente, existen numerosas distribuciones
de probabilidad para medir el tiempo de vida de los componentes. En este trabajo de tesis,
se trabaja con la distribución exponencial debido a que es sencilla su aplicación, y además,
es la más popular en la bibliografía.
