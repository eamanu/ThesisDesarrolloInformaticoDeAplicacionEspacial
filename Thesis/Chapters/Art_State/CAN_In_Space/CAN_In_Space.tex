\section{CAN en la actividad espacial}
CAN ya es utilizado en misiones satelitales, pero en la mayoría de los
casos es utilizado como bus secundario de comunicación, dejando el bus
primario para otros medios de comunicación más clásicos como el
MIL-STD-1553B.

La Agencia Espacial Europea (ESA) es la principal organización que se
esfuerza en desarrollar hardware, firmware y software que implementa
CAN como sistema de comunicación y control a bordo de vehículos espaciales.
Esta organización definió el estándar ECSS-E-ST-15C con el objetivo de
estandarizar el protocolo de comunicación CAN. La ESA observa que existe
una tendencia de producir un cambio de  paradigma 
centralizado, a funciones autónomas distribuidas, además, tiene en cuenta las ventajas que
brinda CAN para el desarrollo de satélites. Por ello, lleva a cabo anualmente
el \textit{CAN in Space Workshop}\footnote{https://indico.esa.int/indico/event/162/}.

Debe destacarse que CAN tiene vasta experiencia de vuelo en aviónica de
vehículos aereos (helicopteros y aviones). Lo cual alienta su aplicación
en la industria espacial.
