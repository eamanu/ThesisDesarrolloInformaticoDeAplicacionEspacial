\section{Métrica y modelado de la confiabilidad de sistemas}
Es de suma importancia llevar a cabo el análisis de la confiabilidad, disponibilidad y mantenibilidad (RAM\footnote{Del ingles, Realibility, Availability, Maintainability) de sistemas satelitales, durante la fase de diseño \citep{Hoque15}. Llevar a cabo esto, es de gran importancia, ya que permite el desarrollo de estrategias que permitan altos grados de confiabilidad, disponibilidad y mantenibilidad \citep{Hoque15}.

Existen dos categorías de medición de la confiabilidad y predición \citep{Schneidewind97}, estas son utilizadas para asegurar la seguridad del software de sistemas críticos \citep{Schneidewind97}, las cuales son:
  \begin{itemize}
    \item medición y predicción que están asociadas con las fallas y errores residuales.
    \item medición y predicción que están asociadas con la disponibilidad del sistema a sobrevir durante la misión sin experimentar fallas en el fallas (o pérdidas) en el sistema.
  \end{itemize}

  Las dos categorías mencionadas anteriormente son explicadas en \cite{Schneidewind97}.

  Según \cite{Liu14} las severidades de las fallas son clasificadas como críticas, peligrosas o triviales, teniendo en cuenta la contribución de esa falla a la pérdida de la misión. Es importante, además, conocer el riesgo de una falla. El riesgo, se define como la posibilidad de que una falla produzca una lesión (por ejemplo, un astronauta en vuelos tripulados), algún daño material (por ejemplo, la destrucción del satélite), o una pérdida (por ejemplo, la pérdida de la misión).

  Dependiendo de la misión, un criterio para definir si un sistema es seguro o no, es reduciendo las fallas que pueden provocar pérdidas de vida, pérdida de la misión o la obligación de abortar una misión \citep{Schneidewind97}. \cite{Schneidewind97} define dos criterios que deben satisfacerse:
  \begin{itemize}
    \item $r(t_t) < r_c$,
    \item $T_F(t_t) > t_m$
  \end{itemize}

  dónde $t_t$ es el Tiempo total de testing (observado o predicto); $r(t_t)$ son las fallas restantes hasta $t_t$; $r_c$ es una valor crítico de fallas restantes ;$T_F(t_t)$ es la métrica para medir el riesgo; y $t_m$ es la misión de la duración.

  Lo anterior signifca que un sistema crítico será seguro si: las fallas restantes en el tiempo de prueba son menores a un valor crítico de cantidad de fallas, o la duración de una misión es mayor al tiempo que se de la siguiente falla.
  
 En la literatura se utilizan modelos matemáticos para modelar los sistemas críticos y calcular así su confiabilidad. La mayoría de ellos asumen, que todas las fallas tienen igual tasa de detección de fallas, como así también la misma severidad, lo cual no es correcto \cite{Liu14}.
