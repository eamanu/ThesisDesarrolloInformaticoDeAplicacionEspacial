\section{Redes distribuidas}\label{sec:redes_distribuidas}
Una red de computadoras hace referencia a un conjunto de computadoras autónomas que se encuentran interconectadas, esto quiere decir que pueden intercambiar información \citep{Tanenbaum12}. Una red distribuida tiene como principal caraceterística que no existe un nodo central que ``gestione'' toda la red. Todas las cargas de las tareas y/o actividades son distribuidas entre los nodos que forman parte de la red.

Una red distribuída es tolerante a fallas si los nodos pueden formar subredes \citep{Stivaros92}. Es decir, la red se debe mantener activa y conectada, con diferentes topologías e interconexiones, que permitan tolerar posibles fallas producidas en algunos nodos y permitir mantener la perfomance \citep{Stivaros92}. Debido a que el procesamiento de la red se encuentra distribuída en todo el sistema, esto brinda una ventaja por encima de los sistemas centralizados desde el punto de vista de la confiabilidad \citep{Pradhan82}. Un componente importante de las redes distribuidas tolerantes a fallas, es la topología del sistema \citep{Pradhan82}.

Siguiendo la notación de \cite{Pradhan82} para describir la topología del sistema se utiliza, un gráfico sin direccionamiento G = <V,E>, donde V representa un set de nodos y E representa un set de relaciones. \cite{Stivaros92} agrega que $V(G) = \{v_1,v_2,v_3, ... v_n \}$ representa un vector de nodos, los cuales tiene probabilidades de operación $P = (p_1, p_2, p_3, p_n)$; y define una función de asignación $\pi$, la cual es una función que asigna V con una probabilidad $P_{\pi(v)}$.

La \ac{FT} de la red G, dado el vector de probabilidades \vec{P} y una función de asignación $\pi$, tal como se viene discutiendo anteriormente, es la probabilidad de que la red continúe funcionando, es decir continúe conectada, aún en la falla (aleatoria) de alguno de sus nodos, esto se denota  como $FT(G;\vec{P},\pi)$.

Se dice que un subset de nodos S, es un estado tolerante a fallas del sistema G, cuando estos nodos se mantienen conectados y funcionales. Se utiliza $\theta$ para indicar el conjunto de todos los S posibles. Un estado tolerante S contribuye $\prod_{v\in S}{P_{\pi (v)}} \prod_{v \notin S} (1-p_{\pi (v)})$ a la probabilidad de \ac{FT}.

Para calcular el total de la \ac{FT} del sistema, incluyendo todo los S, se hace: $$FT(G;\vec{P};\pi) = \sum_{S \in \theta}{Pr(S)} = \sum_{S \in \theta}{\prod_{v\in S}{P_{\pi (v)}} \prod_{v \notin S} (1-p_{\pi (v)})}$$

Una relación entre nodos es representado  como ij, lo cual representa un enlace bidireccional entre nodos. El grado del nodo i representa el número de relaciones que inciden en ese nodo, el cual se escribe $d_i$. Así, $d_i$ está limitado por el número de puertos de entrada y salida disponibles por cada nodo \citep{Pradhan82}. $k_{ij}$ representa el número mínimo de \textit{hop} (hop representa la transmisión a través de un link de datos) \citep{Pradhan82}

\cite{Stivaros92} y \cite{Pradhan82} mencionan la existencia de varias topologías que pueden ser aplicadas en una red distribuida tolerante a fallas. Una topología estrella posee una baja distancia entre nodos, pero una pobre tolerancia a fallas \citep{Pradhan82} \citep{Stivaros92}. La topología anillo, permite un simple ruteo, pero existen grandes distancias internodo. Un sistema completamente interconectado, presenta buenas características tolerantes a fallas, pero tiene un alto costo \citep{Pradhan82}. \cite{Pradhan82} propone una topología para una arquitectura distribuida de comunicación. Esta es una topología robusta, y puede llegar a ser compleja a medida que aumentan los nodos.

\subsection{Algoritmo de ruteo}
Los algoritmos de ruteos son necesarios en el desarollo de una arquitectura tolerante a fallas y reconfigurable. Los algoritmos de ruteo se los pueden dividir en dos: \textit{algoritmo primario} y \textit{algoritmo alternativo}. El primero es utilizado cuando no hay fallas de nodos. Se deben mantener los caminos para llegar, correctamente, al nodo destino. El segundo se utiliza en la presencia de alguna falla, este requiere que sea capaz de detectar la presencia de fallas, para luego reconfigurar el sistema. Estos algoritmos deben ser simples y requerir una mínima cantidad de \ac{HW} y \ac{SW}.

Agregado a lo mencionado en el párrafo anterior, deben existir algoritmos de diagnóstico distribuido de fallas, basados en los algoritmos de ruteo \citep{Pradhan82}.
