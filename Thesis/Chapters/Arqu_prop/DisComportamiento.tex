\section{Modelo dinámico}\label{sec:modelo_dinamico}
En esta sección se modelará el comportamiento de la arquitectura.
A diferencia de lo modelos  que se estudió en la sección anterior (diagramas
de bloques, diagrama de bloques internos) que hacen referencia al
\textit{qué} del sistema, en el modelado del comportamiento se
hace referencia al \textit{cómo} del sistema. Este \textit{cómo}
es descripto por el modelado del comportamiento dentro del
sistema \citep{HoltPery}.

En esta sección se verán  3 diferentes niveles de abstracción
de los modelos de comportamiento del sistema. En primer lugar, el
nivel más alto de abstracción es la máquina de estado, que considera la
interacción entre bloques. El siguiente diagrama que se verá es
el diagrama de secuencia, el cual pone en manifiesto los diferentes
mensajes que son enviados entre las entidades. Y por último, se
desarrolla el diagrama de actividades, siendo este el nivel
más bajo de abstracción del modelado del comportamiento del sistema.

\subsection{Máquina de estado}
En esta subsección se describe el ciclo de vida de los
bloques que conforman la arquitectura. En resúmen, lo que se
pretende demostrar, es el comportamiento de los objetos de la
arquitectura.

En la Figura \ref{fig:StateMachineArqCompleta} se presenta la
máquina de estado de la arquitectura completa.  En esta se
puede observar todos los estados por los cuales la arquitectura
debe recorrer.

\begin{figure}[h!]
 \centering
 \includegraphics[scale=0.4]{images/Capitulo5/StateMachineArqCompleta.JPG}
  \caption{Máquina de estado de la arquitectura completa}
\label{fig:StateMachineArqCompleta}
\end{figure}

A continuación se procede a explicar la máquina de estado. Todo
inicia en el momento en el que la red es alimentada (eléctricamente)
siguiendo los protocolos de CAN \citep{can-ciaWEB}. En primer lugar, la
red pasa a un estado de inicialización de los nodos, en los cuales, estos
son alimentados, se encienden, y comienzan con todos los procedimientos
normales de cualquier sistema embebido. Los cuales son Booteo, checkeo del
funcionamiento y del estado de salud del propio \ac{HW} y los \ac{HW}
conectados a él. Esto incluye la iniciación, del protocolo CANae como
un stack más del servicio del sistema operativo o sistema embebido
que gobierna a cada uno de los nodos. Esto se observa en la Figura
\ref{fig:StateMacineInitNodes}.

\begin{figure}[h!]
 \centering
 \includegraphics[scale=0.4]{images/Capitulo5/InitializeNodes.JPG}
  \caption{Máquina de estado de los nodos.}
\label{fig:StateMacineInitNodes}
\end{figure}

Una vez que todos los nodos se encuentran inicializados, y la red se encuentra
estable\footnote{Defínase como estable, la situación en la que el
  protocolo CANae se inicializó correctamente y está listo para pasar
  a la segunda etapa.}, la arquitectra pasa a un estado de Stand-By.
En este estado, la arquitectura espera por la orden del nodo monitor
para comenzar a configurar la red. Aquí se sigue lo indicado en el
protocolo CANae (Véase apéndice \ref{Appendix:A}). En primer lugar,
se crean las instancias de red localmente. Luego,  el nodo monitor
coordina la creación de las instancias de red remota. Luego se crean
los objetos nodos tanto locales como remotos. Estas actividades
son necesarias para lograr establecer una correcta comunicación
entre nodos utilizando el protocolo CANae (Versión 0.1 Alpha).

Una vez finalizada la creación de la red, la arquitectura completa
pasa a un estado Ready, donde el nodo monitor ya puede ser retirado.
Luego, la arquitectura ignora el nodo monitor (si todavía se encuentra
conectado) y ya puede comenzar a trabajar normalmente.

\subsection{Diagrama de secuencia}
A continuación se muestra los diagramas de secuencias, para entender
cuál es el comportamiento de la arquitectura, en un grado
de abstracción menor.

En la Figura \ref{fig:SecInitArq} se puede observar como en
el momento de alimentar (eléctricamente) a los nodos, este comienza,
en primer lugar, a ejecutar el boostrap. Este bootea al
sistema operativo (o sistema embebido según sea la tecnología
que se utilice). Una vez que el sistema operativo es cargado se
deben ejecutar (preferentemente) dos tipos diferentes de chequeos:
\begin{itemize}
\item Hardware Check
\item Software Check
\end{itemize}

\begin{figure}[h!]
 \centering
 \includegraphics[scale=0.4]{images/Capitulo5/SecuenciaIInitNodes.JPG}
  \caption{Diagrama de secuencia del inicio de la arquitectura.}
\label{fig:SecInitArq}
\end{figure}

Estos son dos piezas de código. El primero tiene como tarea principal
chequear el estado de salud de los equipos de hardware, tanto
del propio nodo, como así también del subsistema que tiene más proximo.
Como se expuso en la subsección \ref{subsec:bridge} cada nodo tiene
próximo a él (físicamente hablando) un subsistema (con su computadora
de vuelo). Por lo tanto el nodo se debe encargar de chequear el
estado de salud de los diferentes componetes de ese subsistema. El
segundo chequeo, se ocupa de verificar el estado de salud del software
tanto del nodo, como del subsistema más próximo.

Luego de que los chequeos se hayan resuelto exitosamente se crea una
instancia del protocolo CANae y da inicio a su servicio NMT. Este
servicio se puede observar en la Figura \ref{fig:NMTC5}. En primer lugar
se debe esperar que el nodo monitor esté inicializado, y este les
envía un mensaje de \textit{create\_remote\_network()} a todos los
nodos conectados a la arquitectura. Este mensaje llega al NMT del
protocolo de CANae de todos los nodos.

\begin{figure}[h!]
 \centering
 \includegraphics[scale=0.4]{images/Secciones/AppendixA/Protocol_NMT.JPG}
  \caption{Diagrama de secuencia de CA}
\label{fig:NMTC5}
\end{figure}

Luego, siguiendo el protocolo de CANae, cada nodo crea su instancia de
nodo. Esto puede observarse en la Figura \ref{fig:CreateNodeNMTC5}.
El objeto de \textit{CANAppLayerController} envía el mensaje \textit{
  create\_node()} al \textit{NodeManagement} de \textit{create\_newtork\_object()} con el objetivo de crear una instancia local de la red.


\begin{figure}[h!]
 \centering
 \includegraphics[scale=0.4]{images/Secciones/AppendixA/create_node.JPG}
  \caption{Crear Objeto Red, Objeto Nodo y Objeto Nodo Remoto}
  \label{fig:CreateNodeNMTC5}
\end{figure}

Luego el nodo monitor envía el mensaje \textit{prepare\_remote\_node()}.
El protocolo CANae prevé los pasos que se muestran en la Figura
\ref{fig:PrepareNodeC5}

\begin{figure}[h!]
 \centering
 \includegraphics[scale=0.4]{images/Secciones/AppendixA/PrepareNode.JPG}
  \caption{Preparar nodo para la conexión a la red CAN}
  \label{fig:PrepareNodeC5}
\end{figure}

Por último, se conectan todo los nodos a la red, para lo cual deben
actualizar el \textit{Node Table Configuration}, y finalmente ya
pueden comenzar a comunicarse. Esto se observan en las Figuras
\ref{fig:StartNodeC52} y \ref{fig:ConnectNodeC5}

\begin{figure}[h!]
 \centering
 \includegraphics[scale=0.4]{images/Secciones/AppendixA/StartNode.JPG}
  \caption{Preparar nodo para la conexión a la red CAN}
  \label{fig:StartNodeC52}
\end{figure}

\begin{figure}[h!]
 \centering
 \includegraphics[scale=0.4]{images/Secciones/AppendixA/connect_node.JPG}
  \caption{Conectar el nodo a la red CAN}
  \label{fig:ConnectNodeC5}
\end{figure}

\subsection{Diagrama de actividades}
En esta subsección se verán algunos diagramas de actividad para modelar
el comportamiento de la arquitectura propuesta. Los diagramas de actividades
comúnmente son utilizados, generalmente, para describir el comportamiento
a muy bajo nivel (a nivel de componentes), en otras palabras para modelar
procesos, mientras que los diagramas de secuencia muestran el comportamiento
entre elementos, y las máquinas de estados  muestran el comportamiento
dentro de los elementos \citep{HoltPery}.

En la Figura \ref{fig:initNodes} se puede observar el proceso, en detalle,
del inicio de los nodos. Todo comienza cuando se alimenta (eléctricamente)
el nodo. En primer lugar, se ejecuta (como ocurre normalmente) el bootstrap,
este bootea al sistema operativo o al sistema embebido (dependiendo de como
se implemente). Los siguientes pasos son los esperados para cualquier sistema
espacial. Se chequea el \ac{HW} propio del nodo, se chequea memoria y componentes.
Luego, se chequea el estado de salud del subsistema. Para ello, se puede pedir
el chequeo a la CPU del subsistema, y este a vez, chequea los componentes
propios del subsistema.

Con estos pasos se puede obtener el Houskeeping (HK) tanto de \ac{HW} propio del nodo
como del subsistema. Este HK es recibido por el sistema operativo
(o sistema embebido), y se comienza a hacer un chequeo del lado del
software. Esto significa que se pueden realizar cualquier tipo de actividad
para chequear que las tareas (o procesos) se ejecuten correctamente.

Para finalizar, el sistema operativo crea una instancia del stack de servicio de
comunicación, dónde se encuentra el protocolo de comunicación CANae.

Debe observarse que el chequeo de \ac{HW} tanto propio como el del
subsistema se encuentra dentro de lo que se denomina \textit{zona interrumpible},
esto significa que si se da alguna \textit{Falla} (Failure en la Figura), debe
activarse el sistema FDIR del nodo. Este, lo que busca en primer lugar,
es detectar la falla, dónde se produjo y por qué. Luego, el sistema trata de
aislar la falla para que no afecte a ningún otro componente. Por último, el
sistema hará el intento de recuperarse de la falla, en el caso de que se
recupere correctamente, se regresa la tarea de chequeo, para
llevar a cabo estas actividades nuevamente; en caso contrario se termina la actividad.

\begin{figure}[h!]
 \centering
 \includegraphics[scale=0.4]{images/Capitulo5/initNodes.JPG}
  \caption{Diagrama de actividades del inicio de nodos}
  \label{fig:initNodes}
\end{figure}

Otra parte importante a modelar de la arquitectura propuesta, es el momento
de la creación de la red CANae. Como se mencionó anteriormente el protocolo
CANae (Véase apéndice \ref{Appendix:A}) requiere de la existencia de un nodo monitor
para coordinar la creación de la red CANae. Este modelo se puede observar
en la Figura \ref{fig:NodeMonitor}.

El nodo monitor inicia la creación y configuración del nodo enviando la
señal \textit{create\_remote\_network()} a todos los nodos conectados a la
red. Cada nodo que recibe la señal, lleva a cabo todas las actividades prevista
por el protocolo CANae para la creación y configuración de la red. Crea las instancias
de la red remota, los objetos red, crea la tabla de configuración de nodos y crea
por último las instancias de los nodos.

\begin{figure}[h!]
 \centering
 \includegraphics[scale=0.4]{images/Capitulo5/NodeMonitor.JPG}
  \caption{Diagrama de actividades del Nodo monitor}
  \label{fig:NodeMonitor}
\end{figure}

Como segundo paso, el nodo monitor envía la señal de \textit{Prepare Remote Node}, la
cual es recibida por todos los nodos y se preparan internamente para comenzar a
comunicarse.

Por último,  el nodo monitor envía la señal para que los nodos se conecten a
la red. Cada nodo se conecta, y actualiza su tabla de configuración del nodo. A partir
de este momento los nodos ya pueden comenzar a enviar mensajes y el nodo monitor ya
puede ser extraído de la red. 

Además, se modela el ``cómo'' se comporta la arquitectura al
momento de tratar de enviar un mensaje de un nodo a otro nodo. Esto se puede
observar en la Figura \ref{fig:SendMessage}. En primer lugar el nodo emisor,
debe ``preparar'' el mensaje. Esto significa que debe empaquetarlo para,
cumplir tanto con los protocolos CANae, como así también con el procolo CAN.

Como segundo paso se debe consultar a la tabla de configuación del nodo para conocer
cuál es el camino óptimo para llegar a entregar el mensaje. Para ello se le aplica algún
algoritmo de ruteo.

Por último, se envía el mensaje através del microcontrolador CAN. Este es recibido por el nodo
destino y luego es procesado. Esto puede observarse, con mayor detalle en el
apéndice \ref{Appendix:A}. 

\begin{figure}[h!]
 \centering
 \includegraphics[scale=0.4]{images/Capitulo5/SendMessage.JPG}
  \caption{Diagrama de actividades del envio de mensajes}
  \label{fig:SendMessage}
\end{figure}
