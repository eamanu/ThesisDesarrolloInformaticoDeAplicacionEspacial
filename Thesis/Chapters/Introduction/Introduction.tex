\chapter{Introducción}\label{chap:intro}
% TODO: Introuducción, extraída desde el plan de tesis,
En el marco del Plan Espacial Nacional, desarrollado por la \ac{CONAE} de Argentina, y con el propósito de llevar a cabo actividades de investigación y 
aplicación, provenientes de la \ac{UNLAM} se presenta este plan de tesis con el fin de ampliar los 
conocimientos y la participación de la \ac{CONAE} y \ac{UNLAM}, en el campo del Desarrollo Informático y 
Ciencias de la Computación.

Las actividades desarrolladas para este trabajo de tesis son realizadas, en su mayor proporción, en 
la Unidad de Desarrollo \ac{INVAP}, ubicada en San Carlos de Bariloche, Provincia de Río Negro. Este 
trabajo se encuentra orientado a brindar un nuevo conocimiento, que ayude en cierta medida, en el 
desarrollo de los diferentes proyectos con los que cuenta actualmente esta empresa, agregando un 
grado de innovación en el resultado que se obtenga.

\ac{INVAP} tiene como visión ser un referente en proyectos tecnológicos a nivel mundial \cite{invapWEB}, 
por lo tanto, debe asegurarse que cada uno de los productos que se lleven a cabo sean competitivos. 
Para lograr cumplir con esto, es necesario que tales proyectos se encuentren a la vanguardia 
tecnológica y científica.  

El desarrollo de proyectos satelitales conlleva costos de importante magnitud, y 
dependen de cada misión. Una parte importante de los costos está conformado por el 
desarrollo\footnote{\textit{Nota: entiéndase por desarrollo al proceso de planificación, análisis, 
diseño e implementación.}} y sobre todo los materiales que se utilizan para su fabricación. Esto 
es debido a que se utilizan componentes que son exclusivos para el ámbito espacial, en otras 
palabras que se encuentran ``calificados para volar''. Estos componentes son fabricados especialmente para soportar el ambiente hostil del espacio.

Si se considera al ámbito espacial como una industria, algo que ha sido demostrado en los últimos 
años; y si se tiene en cuenta las intenciones de crecimiento y competitividad de la empresa INVAP,  de permitir el ingreso de nuestro pais en el mercado satelital \cite{invapWEB}, resulta de gran 
importancia lograr reducir los costos en fabricación y desarrollo de vehículos satelitales.

La \ac{NASA} tiene un enfoque de desarrollo bajo el lema 
``faster, cheaper, better'' \cite{Forsberg99}, lo cual busca desarrollar sus proyectos y misiones 
de foma rápida, barata y mejor. Bajo este enfoque se han realizado diversos estudios e 
investigaciones dando resultados sumamente positivos \cite{Tai99}, \cite{Chau99}, 
\cite{Schneidewind98}, \cite{Forsberg99}. En estos trabajos se utilizan componentes que no se 
encuentran ``calificados para volar'',  los cuales también son llamados componentes \ac{COTS}, o de estantería. Debe mencionarse, que también hubo algunos 
fracasos en su aplicación. 

A simple vista, la utilización de estos componentes ayudaría a reducir costos. Sin embargo, esto 
no es tan directo. Los componentes \ac{COTS} al no estar calificados, se les deben realizar tareas 
de calificación adicional. Además deben ser aplicados a un ambiente, que asegure 
que no fallarán durante la misión; o si fallan, no será motivo de pérdida de la misma. 

Los componentes \ac{COTS} suelen tener un costo de compra entre 100 y 1000 veces menores que aquellos 
que está califcados para volar. Por lo que el aumento en la utilización de estos componentes, 
aplicados al desarrollo de diferentes tipos de satélite, \textbf{permitiría reducir los costos y 
ahorrar algunos millones de dólares del proyecto satelital.} Esto facilitaría el ingreso de 
Argentina en un mercado altamente competitivo.

El desafío de este trabajo de tesis es analizar y estudiar arquitecturas que sean tolerantes a 
fallas, que permitan una correcta comunicación entre los diferentes subsistemas de un vehículo 
espacial de nueva generación, y que tenga como característica principal un cierto grado de confiabilidad, de modo tal que pueda ser aplicado con componentes \ac{COTS}.

\section{Motivación}\label{chap:motivacion}
Los costos de un proyecto satelital se pueden clasificar, a grandes rasgos, en 5 grupos:
\begin{itemize}
 \item Desarrollo
 \item Materiales
 \item Ensamblado, integración, y tests
 \item Lanzamiento
 \item Operaciones
\end{itemize}

Este trabajo de tesis se centrará principalmente en el desarrollo (proceso de planificación, 
análisis, diseño e implementación.), y en los materiales utilizados en la fabricación de vehículos satelitales.

No se puede mencionar a ciencia cierta cuál es el costo “verdadero” de desarrollar un satélite. Este 
depende exclusivamente del tipo de satélite y de la misión. Lo que si se debe tener en claro es que 
las tareas de desarrollo representan una parte muy importante del costo total del proyecto.

Desarrollar un vehículo espacial con componente \ac{COTS}, en un principio podría representar costos 
adicionales, ya que se le deben realizar tareas de calificación adicional, debido a que no están 
“preparados” para resistir las condiciones hostiles del espacio. 

Uno de los puntos positivos, y que motivan la aplicación de componentes COTS, es que a la hora de 
desarrollar varios satélites en base a la misma ingeniería, se puede ahorrar en gran medida en los 
materiales que se utilizan. Los componentes \ac{COTS} suelen tener un costo de compra entre 100 y 1000 
veces menores que aquellos que están calificados para volar. \textbf{Esto ayudaría a ahorrar 
algunos millones de dólares de los proyectos satelitales.}
  
Otra de las ventajas de utilizar componentes \ac{COTS}, es que la mayoría cuentan con una tecnología más 
avanzada que aquellos que son calificados para volar. Esta tecnología permite:
\begin{itemize}
 \item Aumentar prestaciones, mediante el incremento de las capacidades de procesamiento, memoria, 
velocidades de 
procesamiento, etc.
 \item Implementar funciones que son imposibles de aplicar en tecnologías viejas.
 \item Reducir tiempos de desarrollo.
 \item Reducir volumen, masa y consumo
\end{itemize}

El último punto mencionado anteriormente es de especial interés, ya que al reducir volumen y masa, 
permite reducir costos adicionales como el de lanzamiento.

Esta reducción de costos de proyectos satelitales tienen ventajas directas a la hora de introducir a 
Argentina en un mercado altamente competitivo, donde la mínima reducción de estos, representa 
ganancias económicas importantes. 
 
Uno de los puntos en contra de la utilización de componentes \ac{COTS} es que al no ser calificados para 
volar, es necesario llevar a cabo tareas y estrategias inteligentes, con el fin de hacer frente a 
esa “deficiencia”. Por ello, se exige realizar una investigación y análisis de diferentes 
arquitecturas de aviónica, que puedan ser utilizadas para lograr que el sistema sea tolerante a 
fallas, y así, cumplir con los requerimientos de una misión satelital. 

El estudio de arquitecturas tolerantes a fallas, no solamente tiene aplicación en el ámbito 
espacial, si no que también puede ser extendido a cualquier sistema crítico, los cuales necesitan 
ser robustos y tolerantes a fallas, como es el caso de aviones comerciales, plantas nucleares, 
automóviles, etc.

% --------------------- %
% TODO: Hipótesis
% --------------------- %
\section{Hipótesis}
La hipótesis de esta tesis es la siguiente: ``Una arquitectura de aviónica  basadas en componentes 
\ac{COTS}, robusta y tolerante a fallas, es totalmente aplicable y utilizable en vehículos espaciales, 
con un alto nivel de confiabilidad, lo cual permite disminuir la complejidad de los sistemas actuales de aviónica''.

\section{Objetivo del trabajo y preguntas de investigación}

\subsection{Objetivo}
El objetivo de este trabajo es investigar y analizar arquitecturas de comunicación de los 
subsistemas de aviónica tolerante a fallas basada en componentes \ac{COTS} para vehículos 
satelitales de nueva generación.

% secondary objectives
\subsection{Objetivos Específicos}
\begin{enumerate}
 \item Realizar un estudio del estado de la cuestión sobre arquitecturas tolerantes a fallas para 
sistemas críticos.
 \item Investigar y analizar arquitecturas tolerantes a fallas que aseguren la confiabilidad del 
sistema y que sean aplicables en la industria satelital.
 \item Investigar y analizar protocolos de comunicación, para las capas superiores del modelo de 
OSI (modelo de interconexión de sistemas abiertos - ISO/IEC 7498-1), orientados a la tolerancia a 
fallas y confiabilidad de los sistemas. Realizar un estudio comparativo de los diferentes 
protocolos estudiados.
 \item Investigar una metodología para lograr una medición de la tolerancia a fallas en 
arquitecturas de aviónica.
 \item Desarrollar un estudio comparativo de arquitecturas tolerantes a fallas con el fin de obtener 
ventajas y desventajas de cada una de ellas.
 \item Diseñar modelos alternativos de arquitecturas tolerantes a fallas, que tenga un grado de 
confiabilidad tal que permita la aplicación de componentes \ac{COTS}.
 \item Evaluar la confiabilidad de los modelos de arquitecturas (mediante métrica desarrollada en 
este trabajo o siguiendo otras estrategias). 
 \item Proponer el diseño de una nueva arquitectura tolerante a fallas, con un 
grado de confiabilidad suficiente para la aplicación de componentes \ac{COTS} en aviónicas de vehículos 
satelitales.
\item Simular la arquitectura planteada para medir su grado de tolerancia a fallas y perfomance.
\end{enumerate}

\subsection{Preguntas de investigación}\todo[inline]{Ver qué preguntas se lograron responder y mostrarlas en algún apartado}
Para este trabajo de tesis se plantearon las siguientes preguntas de investigación, que se fueron respondiendo a lo largo de este trabajo\:
\begin{itemize}
 \item ¿Es posible la realización de un método de medición del grado de tolerancia a fallas de una arquitectura de aviónica?
 \item ¿Cuál es la estrategia más indicada de tolerancia a fallas que permita brindar un alto grado de confiabilidad en la utilización de componentes
 \ac{COTS} en sistemas críticos?
 \item ¿Cuál es la arquitectura más indicada que permita desarrollar tolerancia a fallas en sistemas críticos basados en componentes \ac{COTS}?
 \item ¿Es factible la utilización de componentes \ac{COTS} en sistemas espaciales?
\end{itemize}
