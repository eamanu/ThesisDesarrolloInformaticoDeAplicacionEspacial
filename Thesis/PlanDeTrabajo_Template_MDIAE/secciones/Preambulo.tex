\textwidth 6in
\topmargin -1.0cm
\textheight = 9in
\oddsidemargin 1cm
\evensidemargin 0cm
\parskip 7.2pt

\usepackage{setspace}
\usepackage{graphicx}
\usepackage{rotating}
\usepackage{latexsym}
\usepackage{t1enc}
\usepackage{times}
\usepackage{mathptmx}
\usepackage{multirow}
\usepackage{lscape}
\ExecuteOptions{dvips}
\usepackage{pstricks}
\usepackage{textcomp}

%para pegar figuras y tablas en lugares fijo con el comando [H]
\usepackage{float}

\usepackage{amsmath}
%\usepackage[pdftex, colorlinks=true]{hyperref}\mu
\usepackage{upgreek}

%Usar el paquete Natbib para un mejor manejo de las citas de bibliografia
%\usepackage{natbib}

%solo para agregar el simbolito de grados: °
\usepackage{gensymb}

%\usepackage{latexsym}
%\usepackage{clrscode} % Pseudocodigo
%\usepackage[Bjornstrup]{Sonny} % Encabezados de capitulos. %Sonny  %Lenny %Bjarne % Glenn %Conny %Rejne %Bjornstrup
\usepackage{titling} % Manejo del titulo.

% paquete fancyhdr para los encabezados y pie de página
%\usepackage{fancyhdr}
%\pagestyle{fancy}

%\DeclareGraphicsExtensions{.png,.jpg,.pdf}
\usepackage{subfigure} %poner dos figuras en al lado
\usepackage[final]{pdfpages} %agregar pdf's al documento


%\usepackage{calc}
%\fancyheadoffset[R]{\marginparsep+\marginparwidth}
\renewcommand{\chaptermark}[1]{\markboth{\thechapter \: \textsc{#1}}{}}
\renewcommand{\sectionmark}[1]{\markboth{\thesection \: \textsc{#1}}{}}
\renewcommand{\subsectionmark}[1]{\markright{\textit{#1}}}
%\fancyhf{}
%\fancyhead[L]{\leftmark}
%\fancyhead[R]{\rightmark}
%\fancyfoot[R]{\thepage}
%\fancypagestyle{plain}{%
%   \fancyhead{} % get rid of headers
%   \renewcommand{\headrulewidth}{0pt} % and the line
%}


%\usepackage{graphics}
\usepackage{amssymb}
\usepackage{amsfonts}
\usepackage{amsmath}
\usepackage{ucs}
\usepackage[spanish]{babel}
\usepackage[utf8x]{inputenc}
\usepackage[T1]{fontenc}
%\usepackage[amsmath,thmmarks]{ntheorem}
%\usepackage{proof}

% paquete para incluir imágenes altas y poder tener texto al lado...
\usepackage{wrapfig}

% paquete para colorear tablas
\usepackage{tabularx,colortbl}
\usepackage{xcolor}
% %   Renombra algunos elementos
%\renewcommand{\contentsname}{Índice}
%\addto\captionsspanish{\renewcommand\contentsname{Indice}}
\renewcommand{\partname}{Parte}
%\addto\captionsspanish{\renewcommand\partname{Parte}}
\renewcommand{\chaptername}{Capítulo}
%\addto\captionsspanish{\renewcommand\chaptername{Capitulo}}
\renewcommand{\appendixname}{Apéndice}
%\addto\captionsspanish{\renewcommand\appendixname{Ap\'ndice}}
\renewcommand{\bibname}{Bibliografía}
%\addto\captionsspanish{\renewcommand\bibname{Referencias}}
\renewcommand{\figurename}{Figura}
%\addto\captionsspanish{\renewcommand\figurename{Figura}}
\renewcommand{\listfigurename}{Índice de figuras}
%\addto\captionsspanish{\renewcommand\listfigurename{\'Indice de figuras}}
\renewcommand{\tablename}{Tabla}
%\addto\captionsspanish{\renewcommand\tablename{Tabla}}
\renewcommand{\listtablename}{Índice de tablas}
%\addto\captionsspanish{\renewcommand\listtablename{\'Indice de Tablas}}
\renewcommand{\contentsname}{Tabla de Contenidos}
%\addto\captionsspanish{\renewcommand\contentsname{Tabla de Contenidos}}
%{\renewcommand\contentsname{Table of Contents}}
