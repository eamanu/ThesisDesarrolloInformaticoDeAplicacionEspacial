\section{Objetivos}
\subsection{Objetivos del trabajo}
\begin{frame}
    \frametitle{Objetivo del trabajo}
    El objetivo de este trabajo es investigar y analizar arquitecturas de comunicación de los subsistemas de aviónica tolerante a fallas basada en componentes COTS para vehículos satelitales de nueva generación.
\end{frame}

\subsection{Objetivos específico}
\begin{frame}[allowframebreaks]
	\frametitle{Objetivo específico}
	\begin{enumerate}
		\item Realizar un estudio del estado de la cuestión sobre arquitecturas tolerantes a fallas para sistemas críticos.
		\item Investigar y analizar arquitecturas tolerantes a fallas que aseguren la confiabilidad del sistema y que sean aplicables en la industria satelital.
		\item Investigar y analizar protocolos de comunicación, para las capas superiores del modelo de OSI (modelo de interconexión de sistemas abiertos - ISO/IEC 7498-1), orientados a la tolerancia a fallas y confiabilidad de los sistemas. 
		\item Investigar una metodología para lograr una medición de la tolerancia a fallas en arquitecturas de aviónica.
		\item Desarrollar un estudio comparativo de arquitecturas tolerantes a fallas con el fin de obtener ventajas y desventajas de cada una de ellas.
		\item Diseñar modelos alternativos de arquitecturas tolerantes a fallas, que tengan un grado de confiabilidad tal, que permita la aplicación de componentes COTS.
		\item Evaluar la confiabilidad de los modelos de arquitecturas.
		\item Proponer el diseño de una nueva arquitectura tolerante a fallas, con un grado de confiabilidad suficiente para la aplicación de componentes COTS en avionicas de vehículos satelitales.
		\item Simular la arquitectura planteada para medir su grado de tolerancia a fallas y performance.
	\end{enumerate}
\end{frame}

\subsection{Preguntas de investigación}
\begin{frame}
	\frametitle{Preguntas de investigación}
	\begin{itemize}
		\item ¿Es posible la realización de un método de medición del grado de tolerancia a fallas de una arquitectura de aviónica?
		\item ¿Cuál es la estrategia más indicada de tolerancia a fallas que permita brindar un alto grado de confiabilidad en la utilización de componentes COTS en sistemas críticos?
		\item ¿Cuál es la arquitectura más indicada que permita desarrollar tolerancia a fallas en sistemas críticos basados en componentes COTS?
		\item ¿Es factible la utilización de componentes COTS en sistemas espaciales?
	\end{itemize}
\end{frame}