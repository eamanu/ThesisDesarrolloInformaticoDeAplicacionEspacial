\chapter*{Resumen} % si no queremos que añada la palabra "Capitulo"
%\addcontentsline{toc}{chapter}{Resumen} % si queremos que aparezca en el índice

El desarrollo de proyectos satelitales conlleva costos de importante magnitud, y 
dependen de cada misión. Estos costos pueden ser clasificados, en forma general, en
5 grupos:

\begin{itemize}
 \item Desarrollo
 \item Materiales
 \item Ensamblado, integración, y tests
 \item Lanzamiento
 \item Operaciones
\end{itemize}

Una parte importante de los costos está conformado por el 
desarrollo, y sobre todo, por los materiales que se utilizan para su fabricación. Esto 
es debido a que se utilizan componentes que son exclusivos para el ámbito espacial, en otras 
palabras que se encuentran ``calificados para volar''. Estos componentes son fabricados especialmente para soportar el ambiente hostil del espacio.

No se puede mencionar a ciencia cierta cuál es el costo “verdadero” de desarrollar un satélite. Este 
depende exclusivamente del tipo de satélite y de la misión. Lo que sí se debe tener en claro es que 
las tareas de desarrollo representan una parte muy importante del costo total del proyecto. Por tal
motivo, este trabajo de tesis se centrará principalmente en el desarrollo (proceso de planificación, 
análisis, diseño e implementación.), y en los materiales utilizados en la fabricación de vehículos satelitales.

El desafío de este trabajo de tesis es analizar y estudiar arquitecturas que sean tolerantes a 
fallas, que permitan una correcta comunicación entre los diferentes subsistemas de un vehículo 
espacial de nueva generación, y que tenga como característica principal un cierto grado de confiabilidad, de modo tal que pueda ser aplicado con componentes \ac{COTS}. Desarrollar un vehículo espacial con componente \ac{COTS}, en un principio podría representar costos 
adicionales, ya que se le deben realizar tareas de calificación adicional, debido a que no están 
“preparados” para resistir las condiciones hostiles del espacio.

Estos componentes COTS tiene varias ventajas sobre los ``clásicos''. Uno de sus puntos positivos,
es que a la hora de desarrollar varios satélites en base a la misma ingeniería, se puede ahorrar en gran medida en los materiales que se utilizan. Los componentes \ac{COTS}
suelen tener un costo de compra hasta 1000 
veces menores que aquellos que están calificados para volar. \textbf{Esto ayudaría a ahorrar 
 millones de dólares de los proyectos satelitales.}

Otra de las ventajas de utilizar componentes \ac{COTS}, es que la mayoría cuentan con una tecnología más avanzada que aquellos que son calificados para volar. Esta tecnología permite:
\begin{itemize}
 \item Aumentar prestaciones, mediante el incremento de las capacidades de procesamiento, memoria, 
velocidades de 
procesamiento, etc.
 \item Implementar funciones que son imposibles de aplicar en tecnologías viejas.
 \item Reducir tiempos de desarrollo.
 \item Reducir volumen, masa y consumo
\end{itemize}

Uno de los puntos en contra de la utilización de componentes \ac{COTS} es que al no ser calificados para volar, es necesario llevar a cabo tareas y estrategias inteligentes, con el fin de hacer frente a 
esa “deficiencia”. Por ello, se exige realizar una investigación y análisis de diferentes 
arquitecturas de aviónica, que puedan ser utilizadas para lograr que el sistema sea tolerante a 
fallas, y así, cumplir con los requerimientos de una misión satelital.

En esta tesis se desarrolla un protocolo de comunicación basada en CAN, que se lo llamó CANae.
Este, es el protocolo utilizado en la arquitectura tolerante a fallas que se propone en este
trabajo, el cual está basado en redes distribuidas. Todos los diseños están orientados a
modelos, y se utiliza el lenguaje de modelado  SysML para su desarrollo y documentación. 
