\section{Conclusiones}
\begin{frame}
	\centering
	\LARGE \textbf{Conclusiones}
\end{frame}

\begin{frame}
	\frametitle{Conclusiones}
	\begin{itemize}
		\item Es factible la utilización de componentes COTS en sistemas espaciales
		\item Se demostró que la estrategia más indicada de tolerancia a fallas, para el desarrollo de sistemas espaciales, es una arquitectura basada en redes distribuida
		\item CANae permite desarrollar tolerancia a fallas en sistemas críticos basados en componentes COTS. 
	\end{itemize}
\end{frame}

\begin{frame}
	\frametitle{Perspectivas a futuro}
	\begin{itemize}
		\item Diseño detallado, desarrollo e  implementación del protocolo CANae
		\item Diseño detallado de la arquitectura propuesta
		\item Estudio de nuevas técnicas de tolerancia a fallas aplicadas a los diferentes niveles de detalle de las arquitecturas de aviónica. 
		\item Desarrollo de algoritmos de ruteo para la distribución de tareas en redes distribuidas
		\item Estudio de tecnología Wireless como medio de comunicación alternativo al cableado
	\end{itemize}
\end{frame}