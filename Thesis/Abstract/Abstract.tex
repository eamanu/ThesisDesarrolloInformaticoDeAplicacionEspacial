\chapter*{Abstract}
\label{chap:abstract}
The space missions development involves costs of major magnitude, the most important costs
being the development, and the materials used for the manufacture of satellites. The high
cost of these materials is because they are manufactured exclusively for their application
to space activity, that mean that they are "qualified to fly." There are other components
called COTS (Commercial Off-The-Shelf). These COTS components are considerably more
economical than the "classic" ones, thus helping to significantly reduce costs. This is
important to the Argentinian space industry. To achieve a correct application of these
components, it is necessary develop  techniques, strategies and architectures that ensure
that the probability of catastrophic failure and degradation are compatible with the mission.
In this work, we firstly studied and proposed models of different topologies that will
be used to develop a fault tolerant architecture based on COTS components. Then, a communication
protocol called CANae is developed. CANae is  based on the CAN (Controller Area Network)
protocol, and CANae is oriented to work in distributed networks. Finally, in this thesis
be proposed an architecture based on both COTS components and distributed networks, that
uses the protocol CANae for the communication with its components. In this way, an
architectural proposal is developed to ensures that the system is reliable, even though
the components that make it up are of low reliability.


\textbf{Keywords: COTS Components, Fault Tolerant Architecture, Avionics, Fault Tolerance, Satellites}
