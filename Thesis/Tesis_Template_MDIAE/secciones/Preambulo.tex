% Configuración de la página: márgenes
%\textwidth 6in
%\textwidth 17cm
%\topmargin -1.5cm
%\textheight = 9in
%\textheight = 25.7cm
%\oddsidemargin 0cm
%\evensidemargin 0cm
% espacio entre párrafos
\parskip 7.2pt
% Paquetes
%configuración de los márgenes
\usepackage[top=2.5cm, bottom=2.5cm, left=2.5cm, right=1.5cm]{geometry}
\usepackage{setspace}




% \textwidth 6in
% \topmargin -1.0cm
% \textheight = 9in
% \oddsidemargin 1cm
% \evensidemargin 0cm
% \parskip 7.2pt

\usepackage{setspace}
\usepackage{graphicx}
\usepackage{rotating}
\usepackage{latexsym}
\usepackage{t1enc}
\usepackage{times}
\usepackage{mathptmx}
\usepackage{multirow}
\usepackage{lscape}
\ExecuteOptions{dvips}
%\usepackage{pstricks}
\usepackage{textcomp}

%Para editar y mandar al pie (con el "bottom") las footnotes
\usepackage[bottom]{footmisc}

% para modificar el caption de las figuras (letra mas chica y en negrita)
\usepackage[footnotesize,bf,center]{caption}

% Eliminar la linea encima de los footnotes
%\renewcommand\footnoterule{\rule{\linewidth}{0pt}}

% Secuencia de 9 símbolos para footnotes (usa asteriscos y otros símbolos en lugar de números)
%\renewcommand{\thefootnote}{\fnsymbol{footnote}}

%para pegar figuras y tablas en lugares fijo con el comando [H]
\usepackage{float}

\usepackage{amsmath}
%\usepackage[pdftex, colorlinks=true]{hyperref}\mu
\usepackage{upgreek}

%Usar el paquete Natbib para un mejor manejo de las citas de bibliografia
\usepackage{natbib}

%\usepackage{plainnat}

%solo para agregar el simbolito de grados: °
\usepackage{gensymb}

%para enumerar las subsubsecciones
\setcounter{secnumdepth}{3}
\setcounter{tocdepth}{3}



%\usepackage{latexsym}
%\usepackage{clrscode} % Pseudocodigo
\usepackage[Sonny]{fncychap} % Encabezados de capitulos. %Sonny  %Lenny %Bjarne % Glenn
\usepackage{titling} % Manejo del titulo.
\usepackage{fancyhdr}
%\DeclareGraphicsExtensions{.png,.jpg,.pdf}
\usepackage[]{subfigure} %poner dos figuras en al lado
\usepackage[final]{pdfpages} %agregar pdf's al documento
\pagestyle{fancy}
\usepackage{enumerate}

%\usepackage{calc}
%\fancyheadoffset[R]{\marginparsep+\marginparwidth}
\renewcommand{\chaptermark}[1]{\markboth{\thechapter \: \textsc{#1}}{}}
\renewcommand{\sectionmark}[1]{\markboth{\thesection \: \textsc{#1}}{}}
\renewcommand{\subsectionmark}[1]{\markright{\textit{#1}}}
\fancyhf{}
\fancyhead[L]{\leftmark}
%\fancyhead[R]{\rightmark}
\fancyfoot[R]{\thepage}
\fancypagestyle{plain}{%
   \fancyhead{} % get rid of headers
   \renewcommand{\headrulewidth}{0pt} % and the line
}


%\usepackage{graphics}
\usepackage{amssymb}
\usepackage{amsfonts}
\usepackage{amsmath}
\usepackage{ucs}
\usepackage[utf8x]{inputenc}
%\usepackage[latin1]{inputenc}
\usepackage[T1]{fontenc}
\usepackage[spanish]{babel}
%\usepackage[ansinew]{inputenc} % Caracteres con acentos.
%\usepackage[amsmath,thmmarks]{ntheorem}
%\usepackage{proof}

% paquete para incluir imágenes altas y poder tener texto al lado...
\usepackage{wrapfig}

% paquete para colorear tablas
\usepackage{tabularx,colortbl}
\usepackage{xcolor}
% %   Renombra algunos elementos
\renewcommand{\contentsname}{Indice}
%\addto\captionsspanish{\renewcommand\contentsname{Indice}}
\renewcommand{\partname}{Parte}
%\addto\captionsspanish{\renewcommand\partname{Parte}}
\renewcommand{\chaptername}{Cap\'itulo}
%\addto\captionsspanish{\renewcommand\chaptername{Capitulo}}
\renewcommand{\appendixname}{Ap\'endice}
%\addto\captionsspanish{\renewcommand\appendixname{Ap\'ndice}}
\renewcommand{\bibname}{Referencias}
%\addto\captionsspanish{\renewcommand\bibname{Referencias}}
\renewcommand{\figurename}{Figura}
%\addto\captionsspanish{\renewcommand\figurename{Figura}}
\renewcommand{\listfigurename}{Indice de figuras}
%\addto\captionsspanish{\renewcommand\listfigurename{\'Indice de figuras}}
\renewcommand{\spanishtablename}{Tabla}
%\addto\captionsspanish{\renewcommand\tablename{Tabla}}
\renewcommand{\spanishlisttablename}{Indice de tablas}
%\addto\captionsspanish{\renewcommand\listtablename{\'Indice de Tablas}}
\renewcommand{\contentsname}{Tabla de Contenidos}
%\addto\captionsspanish{\renewcommand\contentsname{Tabla de Contenidos}}
{\renewcommand\contentsname{Table of Contents}}


%paquete para el manejo de acrónimos

%\usepackage{acronym}
% Generación de acrónimos:   \acrodef{label}[acronym]{written out form}  o \acrodef{label}{written out form} si el acrónimo es igual al label usado
% Uso de los acrónimos:      \ac{label}
% Otro usos del paquete:
\usepackage[printonlyused]{acronym}   %hace que en la lista de acrónimos se muestren solo los utilizados
%\usepackage[printonlyused,withpage]{acronym}  %igual que el anterior pero además muestra la página en la que aparece por primera vez

% definición de colores personalizados para utilización en tablas
\definecolor{celeste}{RGB}{185,211,238}
\definecolor{celeste2}{RGB}{154,192,205}
\definecolor{gris1}{RGB}{224,238,224}
\definecolor{gris2}{RGB}{253,245,230}
\definecolor{gris3}{RGB}{211,211,211}