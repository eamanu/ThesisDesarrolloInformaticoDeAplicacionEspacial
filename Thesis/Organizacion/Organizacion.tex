\chapter*{Organizacion de la tesis}
\label{chap:organizacion}
%\addcontentsline{toc}{chapter}{Agradecimientos}

Este trabajo de tesis está organizado de manera tal que, el lector logre comprender
cuál es la problemática que se plantea el autor, conozca el sustento
teórico sobre los cuales la tesis se apoya, y con ello,  y de manera natural, se logre
entender el significado y la importancia de la propuesta de la arquitectura diseñada
en esta tesis.

En el Capítulo \ref{chap:intro} (página \pageref{chap:intro}) se centra en que el lector
entienda la motivación (Sección \ref{chap:motivacion} página \pageref{chap:motivacion})
que impulsan este trabajo de investigación y desarrollo, como así también las
hipótesis (Sección \ref{sec:hipotesis} página \pageref{sec:hipotesis}), objetivos
(Sección \ref{sec:objetivos} página \pageref{sec:objetivos}), objetivos específicos
(Sección \ref{sec:objetivos_especificos} página \pageref{sec:objetivos_especificos}) y las
preguntas de investigación (\ref{sec:preguntas_investigacion} página \pageref{sec:preguntas_investigacion}) que el autor se propone resolver.

En el Capítulo \ref{chap:marco_teorico} (página \pageref{chap:marco_teorico})
En este capítulo se introduce al lector a la teoría en la cual se basa este trabajo de tesis. Entre
las secciones más importantes que se verán se puede mencionar, que en la
Sección \ref{sec:terminologia} (página \pageref{sec:terminologia}) se explica brevemente la
terminología que se utiliza a lo largo de este trabajo. Es importante leer esta sección ya que
algunos términos pueden considerarse sinónimos en el habla común, pero en esta tesis, tienen
significados bien diferenciados.
En la Sección \ref{sec:fiabilidad_software} (página \pageref{sec:fiabilidad_software}) explica
qué es la fiabilidad aplicada en sistemas (de software) y cómo se puede clasificar la fiabilidad. En
Sección \ref{sec:impedimentos} (página \pageref{sec:impedimentos}) se detalla el concepto de los
impedimentos de la fiabiliad, cuáles son los orígenes de las fallas, los modos comúnes de fallas
y las fallas en el software.
La tolerancia a fallas, el núcleo de este trabajo, se explica en la Sección \ref{chap:FaultTolerance}
(página \pageref{chap:FaultTolerance}). En esta sección se detalla el signficado de la tolerancia
fallas, importante para entender este trabajo de tesis.
En la Sección \ref{sec:protocolos_comunicacion} (página \pageref{sec:protocolos_comunicacion})
se brinda un resúmen teórico de diferentes protocolos de comunicación existentes en la actualidad.
En la Sección \ref{seccion:ProtocoloCAN} (página \pageref{seccion:ProtocoloCAN}) se detalla el
protocolo CAN que es utilizado como principal protocolo de comunicación. La arquitectura propuesta
en este trabajo de tesis utiliza un protocolo de comunicación basado en CAN. En esta sección se describe capa física (página \pageref{subsec:capafisca}), capa de enlace (página \pageref{subsec:capa_enlace}), formato del mensaje (página \pageref{subsec:formato_mensaje}) y se comenta sobre un protocolo basado en CAN que es utilizado en aplicaciones de aeronáutica llamado CANAerospace (página \pageref{subsec:CANaerospace})

En el Capítulo \ref{chap:estado_del_arte} (página \pageref{chap:estado_del_arte})


