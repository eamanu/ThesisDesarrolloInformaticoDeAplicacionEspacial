\section{DBT}\label{Appendix:DBT}
La principal complicación del desarrollo de una red distribuída basada en el BUS
CAN, es que puedan comunicarse correctamente entre sí, cómo se deben asignar
los identificadores (Node-ID) correctamente a cada uno de los nodos y cómo se
asignan los tiempos de inhibición. Los Node-ID y los tiempos de inhibición se
deben distribuir entre los nodos de tal manera de asegurar que:
\begin{itemize}
\item Se preveean los conflictos entre nodos. Por ejemplo diferentes funciones
  que usan el mismo identificador.
\item Se preveean la falta de coincidencia. Por ejemplo que existan diferentes
  indentificadores para el mismo nodo.
\item ofrecer el control integrado para un comportamiento dinámico del
  sistema.  
\end{itemize}

El protocolo CAN (CAN Application Layer for Industrial Applications
CiA/DS204-1) dispone la existencia de 3 métodos para distribuir identificadores
y tiempo de inhibición a un módulo.

\begin{itemize}
\item \textbf{Distribución estándar}: en este método los identificadores y
  tiempos de inhibición son estandarizados por el módulo proveedor (nodo
  monitor. Una distribución estándar require la estandarización de todas las
  funciones y su correspondiente identificador.
\item \textbf{Distibución estática}: En este método los identificadores y los
  tiempos de inhibición son fijos. Todos los identificadores y los tiempos de
  inhibición son fijados en el momento del desarrollo.
\item \textbf{Distribución dinámica}: En este método los identificadores y los
  tiempos de inhibición son distribuidos vía red CAN a través del servicio
  estándar y un protocolo definido.  
\end{itemize}

En esta instancia de trabajo, para el desarrollo del protocolo CANae, se utilizó
el \textit{método de distribución estática}. Esto exige que los nodos ya cuenten
con un identificador y tiempos de inhibición por defecto. Esto se asigna en el
momento del desarrollo del diseño e ingeniería de la arquitectura de aviónica.
Esto reduce considerablemente la complejidad de la entidad DBT. Si bien
originalmente, el DBT es un elemento de servicio de la capa de aplicación CAN
que ofrece una distribución dinámica de identificadores y tiempos de inhibición,
en CANae es utilizado para asegurar la correcta comunicación y consistencia de
los nodos y la red en su conjunto.

\subsection{Objetos y servicios de DBT}
El DBT de CANae utiliza 2 objetos para modelar su funcionamiento:
\begin{itemize}
\item \textbf{Node-ID Database (NodeTableConfiguration)}: Esta tabla contiene la
  definición de todos los nodos conectados a la red. Esta tabla está compuesta
  por Node-ID Definitions
\item \textbf{Node-ID Definition}: define todos los atributos de un Node-ID.
  Contiene una definición de usuario creado por el usuario.
\end{itemize}

El DBT de CANae ofrece las siguientes categorias de servicio:
\begin{itemize}
\item \textbf{Distribution Control Services}: esta es tarea del nodo monitor de
  enviar el identificador y los tiempos de inhibición a todos los nodos
  conectados a la red. Los datos que son enviados por el nodo monitor son
  corroborados por los nodos, comprobando que el ID enviado coincida con su
  propio Node-ID.
\item \textbf{Consistency Control Services}: através de este servicio cada nodo
  puede detectar inconsistencia en el NodeTableConfiguration e inconsistencia
  entre nodos.
\end{itemize}

\subsection{Descripción de los servicios DBT}
Los servicios se describen en forma de tabla que contiene los parámetros de
cada función que se define para ese servicio. Los parámetros determinan el tipo
de servicio.  Todos los servicios asumen que no ocurrieron ningún tipo de error
en la capa de física ni en la capa de enlace de datos.

\subsection{Objetos DBT}
Los objetos que utiliza la entidad DBT para modelar el servicio son los
siguientes:
\begin{itemize}
\item \textbf{Node Table Configuration (Node-ID Database)}
  \begin{itemize}
  \item \textbf{Atributos}
    \begin{itemize}
    \item \underline{Estado}: Uno de los valores {ENABLED, DISABLED}. Este
      atributo indica si el DBT Monitor es capaz de distribuir los Node-ID
      y tiempos de inhibición. En esta instancia de trabajo el valor del estado
      en el DBT del Nodo Monitor será DSIABLED
      \item \underline{Node Definition Set} set de todas las definciones. 
    \end{itemize}
  \end{itemize}

\item \textbf{Node-ID Defintion}:
  Las definciones contiene los siguientes datos:
  \begin{itemize}
  \item \underline{ID}: Valor en el rango de [0,255]
  \item \underline{Mínimo tiempo de inhibición}: en el rango [0,65535] indicando
    el valor mínimo en unidades de 100$\mu$sec para el tiempo de inhibición.
  \end{itemize}
  
\end{itemize}

\subsection{Servicios DBT}
\subsubsection{Distribution Control Services}
\begin{itemize}
  \item create\_node\_table\_configuration()
  \item enable\_distribution()
  \item disable\_distribution()
  \item create\_node\_definition(UInteger lNode-ID, UInteger hNode-ID, UInteger minInhibitTime)
    \begin{itemize}
    \item \textbf{UInteger lNode-ID}: Límite inferior del rango de IDs.
    \item \textbf{UInteger hNode-ID}: Límite superior del rango de IDs.
    \item \textbf{UInteger minInhibitTime}: Tiempo mínimo de inhibición de cada
      nodo.
    \end{itemize}
  \item delete\_node\_definition(UInteger lNode-ID, UInteger hNode-ID)
\end{itemize}

\subsubsection{Consistency Control Service}

\begin{itemize}
\item get\_checksum(UInteger Node-ID, Float checksum)
  \begin{itemize}
      \item \textbf{UInteger Node-ID}: ID del nodo que desea controlar la
    consistencia.
      \item \textbf{Float checksum}: checksum de la DBT Definition.
  \end{itemize}
\end{itemize}

